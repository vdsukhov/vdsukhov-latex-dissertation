
{\actuality} Задача ранжирования набора данных по некоторым заданным метрикам является одной из приоритетных задач анализа и обработки информации.
Существует множество областей и подходов, направленных на решение этих задач.
Однако, не всегда методы, которые были разработаны для определенных математических моделей, могут быть экстраполированы на другие.
Поэтому, остаются научные области, которые требуют развития принципиально новых подходов для решения задачи ранжирования и выделения релевантной информации.

% Подводка к существующим проблемам:
% 1) В целом рост числа экспериментов
% 2) Усложнение дизайна

Примером такой научной области является системная биология, где ежегодно наблюдается прирост проводимых экспериментов по анализу организации и взаимодействия генов.
Также такие эксперименты принято называть экспериментами по анализу \textit{экспрессии генов}.
При этом ситуация не ограничивается простым ростом проводимых экспериментов, а также наблюдается тенденция их усложнения.
В свою очередь, это выражается усложнением структур данных, которые описывают эти эксперименты.

Классическим подходом анализа экспериментов экспрессии генов является применение методов математической статистики и вычислительных алгоритмов.
Этот подход направлен на исследование того, насколько не случайны некоторые свойства набора генов, который по сути своей является лишь подмножеством от исходной структуры данных.


Зачастую исследователи предоставляют открытый доступ к результатам своих экспериментов.
Для систематизации таких данных были разработаны специальные публично открытые репозитории.
Такие репозитории продолжают содержать в себе потенциально интересную информацию, которая ещё не была обнаружена и проанализирована до этого.

Повторное использование и анализ результатов экспериментов позволяет сразу решить несколько задач.
Во-первых, это открывает исследователям возможность находить схожие эксперименты в терминах экспрессии генов, которые в своей основе имеют различные биологические предпосылки.
Как результат, такие ситуации позволяют формулировать новые биологические гипотезы, затем валидировать их путем проведения дополнительных экспериментов и тем самым раскрывать новые биологические закономерности.
Во-вторых, использование вычислительных методов, направленных на повторный анализ существующей информации, оказывается экономически выгоднее, нежели прямолинейное проведение биологического эксперимента.



Несмотря на наличие методов, которые производят анализ экспрессии генов, большинство из них разрабатывались не для решения задачи ранжирования открытых репозиториев по некоторым метрикам релевантности.
С другой стороны, множество методов имеют существенный недостаток, который выражается в наличии некоторых ограничений на структуру исследуемых данных.

Таким образом, разработка и исследование системы поиска релевантных биологических экспериментов является актуальной задачей в области системной биологии и области анализа информации. 
Разработка и применение поисковых систем потенциально позволит формулировать и обнаруживать новые биологические механизмы, лежащие в основе процессов, протекающих в живых организмах.

% \ifsynopsis
% Этот абзац появляется только в~автореферате.
% Для формирования блоков, которые будут обрабатываться только в~автореферате,
% заведена проверка условия \verb!\!\verb!ifsynopsis!.
% Значение условия задаётся в~основном файле документа (\verb!synopsis.tex! для
% автореферата).
% \else
% Этот абзац появляется только в~диссертации.
% Через проверку условия \verb!\!\verb!ifsynopsis!, задаваемого в~основном файле
% документа (\verb!dissertation.tex! для диссертации), можно сделать новую
% команду, обеспечивающую появление цитаты в~диссертации, но~не~в~автореферате.
% \fi

% {\progress}
% Этот раздел должен быть отдельным структурным элементом по
% ГОСТ, но он, как правило, включается в описание актуальности
% темы. Нужен он отдельным структурынм элемементом или нет ---
% смотрите другие диссертации вашего совета, скорее всего не нужен.


{\progress} Существует ряд методов, которые решают задачу анализа экспрессии генов в рамках одного биологического эксперимента.
Одним из популярных и широко используемых методов является Gene Set Enrichment Analysis (GSEA).
В выпускной квалификационной работе \cite{KorotkevichVKR} был рассмотрен прототип метода вычисления сколь угодно малых P-значений с хорошей относительной точностью, что являлось основным недостатком оригинального метода GSEA.
Этот метод позволяет быстро вычислять сколь угодно малые P-значения и основан на применении многоуровневого метода Монте-Карло.
Однако в работе не было досконально изучена накапливаемая ошибка и степень сходимости метода для различных практических приложений.

{\aim} данной работы является создание эффективного метода поиска релевантных биологических экспериментов по заданному набору генов.

Для~достижения поставленной цели необходимо было решить следующие {\tasks}:
\begin{enumerate}[beginpenalty=10000] % https://tex.stackexchange.com/a/476052/104425
  \item Разработка метода FGSEA-Multilevel на базе существующего прототипа для анализа представленности функциональных наборов генов на основе применения многоуровневого метода Монте-Карло.
  \item Разработка метода анализа статистической значимости наборов генов, который не требует наличия явной аннотации эксперимента. 
  \item Разработка системы поиска релевантных биологических экспериментов по заданному набору генов среди базы данных публично доступных экспериментов анализа экспрессии генов.
\end{enumerate}


{\novelty}
\begin{enumerate}[beginpenalty=10000] % https://tex.stackexchange.com/a/476052/104425
  \item Впервые предложена оригинальная статистика для набора генов, которая не требует наличия явной аннотации биологического эксперимента.
  \item Впервые разработан оригинальный метод анализа статистической значимости  набора генов, который позволяет осуществлять анализ экспериментов без наличия явной аннотации биологического эксперимента.
  \item Реализована оригинальная поисковая система для нахождения релевантных биологических экспериментов по заданому набору генов.
\end{enumerate}

{\defpositions}
\begin{enumerate}[beginpenalty=10000] % https://tex.stackexchange.com/a/476052/104425
  \item Разработан метод FGSEA-Multilevel для анализа статистической значимости функциональных наборов генов, основанный на применении многоуровневого метода Монте-Карло.
  \item Разработан новый метод Gene Set Co-Regulation Analysis (GESECA), который позволяет в автоматическом режиме проводить анализ статистической значимости для экспериментов без требования наличия явной аннотации эксперимента.
  \item Разработана новая система поиска биологических экспериментов, которая принимает на вход произвольный набор генов и возвращает ранжированный список биологических экспериментов.
\end{enumerate}



{\passport} Работа соответствует паспорту специальности 05.13.17 - <<Теоретические основы информатики>> и относится к пункту 9: <<Разработка новых интернет-технологий, включая средства поиска, анализа и фильтрации информации, средства приобретения знаний и создания онтологии, средства интеллектуализации бизнес-процессов.>>.

{\theorinfluence} состоит в постановке задачи анализа статистической значимости наборов генов для экспериментов, не имеющих явную аннотацию.

{\influence} состоит в том, что реализованные методы могут быть использованы для упрощения процесса анализа и исследования биологических процессов, протекающих в живых организмах. 
Также они позволяют исследователям синтезировать абсолютно новые предположения о биологических закономерностях, и находить подтверждения для гипотез, сформированных раннее. 

{\methods} В работе используются методы анализа дифференциальной экспрессии генов, методы анализа представленности функциональных наборов генов.


% В папке Documents можно ознакомиться с решением совета из Томского~ГУ
% (в~файле \verb+Def_positions.pdf+), где обоснованно даются рекомендации
% по~формулировкам защищаемых положений.

{\reliability} полученных результатов обеспечивается корректно поставленными задачами, а также большим числом проведенных численных экспериментов.
Полученные результаты находятся в хорошем соответствии с результатами других ислледователей, которые были получены при определенных ограничениях на свойствах исследуемых ими объектах.

{\integration} Разработанный метод FGSEA-Multilevel был реализован и внедрен в пакет fgsea, написанный на языках программирования R и C++. 
Данный пакет является свободно доступным и размещен в архиве биологических пакетов Bioconductor (\url{https://www.bioconductor.org/}).
При этом пакет fgsea входит в число наиболее популярных пакетов из всех доступных пакетов в архиве.
Так в настоящее время ранг пакета по числу обращений и скачиваний равен 38-ми из более чем 2000 различных пакетов, представленных в Bioconductor.
Разработанный метод GESECA также является свободно доступным и может быть скачан с github-репозитория по адресу \url{https://github.com/ctlab/fgsea/tree/geseca}. Поисковая система релевантных биологических экспериментов по заданному набору генов реализована в виде веб-сервиса и доступна по адресу \url{https://ctlab.itmo.ru/geseca/}. Все разработанные методы активно применяются в процессе подготовки магистров по программе <<Биоинформатика и системная биология>> в Университете ИТМО.


{\probation}
Основные результаты работы докладывались на:
\begin{enumerate}
    \item Всероссийский конгресс молодых ученых Университета ИТМО, 2019 г., Университет ИТМО, Санкт-Петербург.
    \item Всероссийская научная конференция по проблемам информатики СПИСОК-2019. 2019 г., СПбГУ, Санкт-Петербург.
    \item Moscow Conference on Computational Molecular Biology (MCCMB 2019), 2019 г., Москва.
    \item Всероссийский конгресс молодых ученых Университета ИТМО, 2020 г., Университет ИТМО, Санкт-Петербург
    \item CSHL Meeting on Biological Data Science 2020,  2020 г., онлайн конференция.
    \item Moscow Conference on Computational Molecular Biology (MCCMB 2021). 2021 г., Москва.
    \item Научная и учебно-методическая конференция Университета ИТМО, 2022 г., Университет ИТМО, Санкт-Петербург.
\end{enumerate}

{\contribution} Автор принимал активное участие в процессе анализа и совершенствования метода FGSEA-Multilevel. 
Автором были определены требуемые параметры для сходимости метода при приложениях его для практических задач. 
Был осуществлен анализ теоретических свойств распределения P-значений и совершен корректный переход к вычислениям логарифмов P-значений в методе FGSEA-Multilevel.
Автор принимал активное участие во всех этапах, связанных с разработкой и реализацией метода GESECA, совместно с Сергушичевым~А.А. и Артемовым~М.Н.
Диссертантом была полностью написана программная реализация метода GESECA.
Проведены валидация и верификация корректности работы метода на практике.
Автором разработан веб-сервис, который служит поисковой системой для поиска релевантных экспериментов по заданному набору генов.

\ifnumequal{\value{bibliosel}}{0}
{%%% Встроенная реализация с загрузкой файла через движок bibtex8. (При желании, внутри можно использовать обычные ссылки, наподобие `\cite{vakbib1,vakbib2}`).
    {\publications} Основные результаты по теме диссертации изложены
    в~XX~печатных изданиях,
    X из которых изданы в журналах, рекомендованных ВАК,
    X "--- в тезисах докладов.
}%
{%%% Реализация пакетом biblatex через движок biber
    \begin{refsection}[bl-author, bl-registered]
        % Это refsection=1.
        % Процитированные здесь работы:
        %  * подсчитываются, для автоматического составления фразы "Основные результаты ..."
        %  * попадают в авторскую библиографию, при usefootcite==0 и стиле `\insertbiblioauthor` или `\insertbiblioauthorgrouped`
        %  * нумеруются там в зависимости от порядка команд `\printbibliography` в этом разделе.
        %  * при использовании `\insertbiblioauthorgrouped`, порядок команд `\printbibliography` в нём должен быть тем же (см. biblio/biblatex.tex)
        %
        % Невидимый библиографический список для подсчёта количества публикаций:
        \printbibliography[heading=nobibheading, section=1, env=countauthorvak,          keyword=biblioauthorvak]%
        \printbibliography[heading=nobibheading, section=1, env=countauthorwos,          keyword=biblioauthorwos]%
        \printbibliography[heading=nobibheading, section=1, env=countauthorscopus,       keyword=biblioauthorscopus]%
        \printbibliography[heading=nobibheading, section=1, env=countauthorconf,         keyword=biblioauthorconf]%
        \printbibliography[heading=nobibheading, section=1, env=countauthorother,        keyword=biblioauthorother]%
        \printbibliography[heading=nobibheading, section=1, env=countregistered,         keyword=biblioregistered]%
        \printbibliography[heading=nobibheading, section=1, env=countauthorpatent,       keyword=biblioauthorpatent]%
        \printbibliography[heading=nobibheading, section=1, env=countauthorprogram,      keyword=biblioauthorprogram]%
        \printbibliography[heading=nobibheading, section=1, env=countauthor,             keyword=biblioauthor]%
        \printbibliography[heading=nobibheading, section=1, env=countauthorvakscopuswos, filter=vakscopuswos]%
        \printbibliography[heading=nobibheading, section=1, env=countauthorscopuswos,    filter=scopuswos]%
        %
        \nocite{*}%
        %
        {\publications} Основные результаты по теме диссертации изложены в~\arabic{citeauthor}~печатных изданиях,
        \arabic{citeauthorvak} из которых изданы в журналах, рекомендованных ВАК\sloppy%
        \ifnum \value{citeauthorscopuswos}>0%
            , \arabic{citeauthorscopuswos} "--- в~периодических научных журналах, индексируемых Web of~Science и Scopus\sloppy%
        \fi%
        \ifnum \value{citeauthorconf}>0%
            , \arabic{citeauthorconf} "--- в~тезисах докладов.
        \else%
            .
        \fi%
        \ifnum \value{citeregistered}=1%
            \ifnum \value{citeauthorpatent}=1%
                Зарегистрирован \arabic{citeauthorpatent} патент.
            \fi%
            \ifnum \value{citeauthorprogram}=1%
                Зарегистрирована \arabic{citeauthorprogram} программа для ЭВМ.
            \fi%
        \fi%
        \ifnum \value{citeregistered}>1%
            Зарегистрированы\ %
            \ifnum \value{citeauthorpatent}>0%
            \formbytotal{citeauthorpatent}{патент}{}{а}{}\sloppy%
            \ifnum \value{citeauthorprogram}=0 . \else \ и~\fi%
            \fi%
            \ifnum \value{citeauthorprogram}>0%
            \formbytotal{citeauthorprogram}{программ}{а}{ы}{} для ЭВМ.
            \fi%
        \fi%
        % К публикациям, в которых излагаются основные научные результаты диссертации на соискание учёной
        % степени, в рецензируемых изданиях приравниваются патенты на изобретения, патенты (свидетельства) на
        % полезную модель, патенты на промышленный образец, патенты на селекционные достижения, свидетельства
        % на программу для электронных вычислительных машин, базу данных, топологию интегральных микросхем,
        % зарегистрированные в установленном порядке.(в ред. Постановления Правительства РФ от 21.04.2016 N 335)
    \end{refsection}%
    \begin{refsection}[bl-author, bl-registered]
        % Это refsection=2.
        % Процитированные здесь работы:
        %  * попадают в авторскую библиографию, при usefootcite==0 и стиле `\insertbiblioauthorimportant`.
        %  * ни на что не влияют в противном случае
    \end{refsection}%
        %
        % Всё, что вне этих двух refsection, это refsection=0,
        %  * для диссертации - это нормальные ссылки, попадающие в обычную библиографию
        %  * для автореферата:
        %     * при usefootcite==0, ссылка корректно сработает только для источника из `external.bib`. Для своих работ --- напечатает "[0]" (и даже Warning не вылезет).
        %     * при usefootcite==1, ссылка сработает нормально. В авторской библиографии будут только процитированные в refsection=0 работы.
}

% При использовании пакета \verb!biblatex! будут подсчитаны все работы, добавленные
% в файл \verb!biblio/author.bib!. Для правильного подсчёта работ в~различных
% системах цитирования требуется использовать поля:
% \begin{itemize}
%         \item \texttt{authorvak} если публикация индексирована ВАК,
%         \item \texttt{authorscopus} если публикация индексирована Scopus,
%         \item \texttt{authorwos} если публикация индексирована Web of Science,
%         \item \texttt{authorconf} для докладов конференций,
%         \item \texttt{authorpatent} для патентов,
%         \item \texttt{authorprogram} для зарегистрированных программ для ЭВМ,
%         \item \texttt{authorother} для других публикаций.
% \end{itemize}
% Для подсчёта используются счётчики:
% \begin{itemize}
%         \item \texttt{citeauthorvak} для работ, индексируемых ВАК,
%         \item \texttt{citeauthorscopus} для работ, индексируемых Scopus,
%         \item \texttt{citeauthorwos} для работ, индексируемых Web of Science,
%         \item \texttt{citeauthorvakscopuswos} для работ, индексируемых одной из трёх баз,
%         \item \texttt{citeauthorscopuswos} для работ, индексируемых Scopus или Web of~Science,
%         \item \texttt{citeauthorconf} для докладов на конференциях,
%         \item \texttt{citeauthorother} для остальных работ,
%         \item \texttt{citeauthorpatent} для патентов,
%         \item \texttt{citeauthorprogram} для зарегистрированных программ для ЭВМ,
%         \item \texttt{citeauthor} для суммарного количества работ.
% \end{itemize}
% % Счётчик \texttt{citeexternal} используется для подсчёта процитированных публикаций;
% % \texttt{citeregistered} "--- для подсчёта суммарного количества патентов и программ для ЭВМ.

% Для добавления в список публикаций автора работ, которые не были процитированы в
% автореферате, требуется их~перечислить с использованием команды \verb!\nocite! в
% \verb!Synopsis/content.tex!.
