
{\actuality} Разработка алгоритмов обнаружения совместной регуляции экспрессии генов является одной из приоритетных областей в системной биологии.
Связано это с тем, что такие методы позволяют исследователям обнаруживать новые биологические закономерности, которые описывают процессы протекающие в клетках живых организмов, и подтверждать ранее сформированные гипотезы.

% Подводка к существующим проблемам:
% 1) В целом рост числа экспериментов
% 2) Усложнение дизайна

В настоящее время существует большое число различных алгоритмов \todo{\autocite{}}, которые направлены на решение задачи поиска совместно экспрессирующихся генов. 
Однако, можно выделить две основные проблемы, которые требуют развития альтернативных подходов для их решения.

Во-первых, наблюдается стремительный рост числа публикаций экспериментов транскриптомных данных.
В свою очередь, это требует от исследователей реализации эффективных и быстрых алгоритмов по обнаружению совместной экспрессии генов, которые бы позволяли использовать их в автоматической, рутинной манере для последовательной обработки большого числа экспериментов.

Во-вторых, наряду с ростом числа экспериментов наблюдается тенденция усложнения дизайнов экспериментов.
Отличительной особенностью экспериментов со сложным дизайном является наличие большого числа групп биологических образцов, где каждая группа соответствует определенному биологическому состояния.
Для анализа таких экспериментов требуется развитие принципиально новых алгоритмов, которые позволят осуществлять их обработку. 

% Вывод о том, что всё вышесказанное требует развития новых методов
Таким образом, разработка алгоритмов для обнаружения совместной регуляции экспрессии генов в транскриптомных данных является актуальной задачей в области системаной биологии. 
Разработка и применение этих алгоритмов потенциально позволит формулировать и обнаружать новые биологические механизмы, лежащие в основе процессов протекающих в живых организмах. 

% \ifsynopsis
% Этот абзац появляется только в~автореферате.
% Для формирования блоков, которые будут обрабатываться только в~автореферате,
% заведена проверка условия \verb!\!\verb!ifsynopsis!.
% Значение условия задаётся в~основном файле документа (\verb!synopsis.tex! для
% автореферата).
% \else
% Этот абзац появляется только в~диссертации.
% Через проверку условия \verb!\!\verb!ifsynopsis!, задаваемого в~основном файле
% документа (\verb!dissertation.tex! для диссертации), можно сделать новую
% команду, обеспечивающую появление цитаты в~диссертации, но~не~в~автореферате.
% \fi

% {\progress}
% Этот раздел должен быть отдельным структурным элементом по
% ГОСТ, но он, как правило, включается в описание актуальности
% темы. Нужен он отдельным структурынм элемементом или нет ---
% смотрите другие диссертации вашего совета, скорее всего не нужен.

\fixme{
{\progress}\ Раннее в выпускной квалификационной работе \todo{\autocite{}}
}

{\aim} данной работы является создание эффективных методов для обнаружения закономерностей в транскриптомных данных в виде совместной регуляции экспрессии наборов генов.

Для~достижения поставленной цели необходимо было решить следующие {\tasks}:
\begin{enumerate}[beginpenalty=10000] % https://tex.stackexchange.com/a/476052/104425
  \item \fixme{Разработка метода анализа представленности функциональных наборов генов на основе применения многоуровневого метода Монте-Карло.}
  \item Разработка метода анализа представленности наборов генов в транскрпитомных данных со сложным дизайном эксперимента.
  \item Разработка системы поиска биологических экспериментов с совместной регуляцией для заданных генов в публичных базах транскриптомных данных.
\end{enumerate}


{\novelty}
\begin{enumerate}[beginpenalty=10000] % https://tex.stackexchange.com/a/476052/104425
  \item Впервые представлена оригинальная статистика представленности набора генов для транскриптомных данных со сложным дизайном биологических экспериментов.
  \item Впервые разработан оригинальный метод анализа представленности набора генов, который позволяет осуществлять анализ транскриптомных данных со сложным дизайном эксперимента. 
  \item Реализована оригинальная поисковая система для нахождения релевантных биологических экспериментов по заданому набору генов. Данная система способна работать с любыми видами технологий получения транскриптомных данных, которые возвращают результаты в виде числовых матриц.
\end{enumerate}

{\defpositions}
\begin{enumerate}[beginpenalty=10000] % https://tex.stackexchange.com/a/476052/104425
  \item \fixme{Разработан новый метод FGSEA-Multilevel для анализа представленности функциональных наборов генов, основанный на применении многоуровневого метода Монте-Карло.}
  \item Разработан новый метод Gene Set Co-Regulation Analysis (GESECA), который позволяет  в автоматическом режиме проводить анализ представленности для транскриптомных данных со сложными дизайнами экспериментов.
  \item Разработана новая система поиска биологических экспериментов, которая принимает на вход произвольный набор генов и возвращает ранжированный список биологических экспериментов.
\end{enumerate}



{\passport} Работа соответствует паспорту специальности 05.13.17 - <<Теоретические основы информатики>> и относится к пункту 5: <<Разработка и исследование моделей и алгоритмов анализа данных, обнаружения закономерностей в данных и их извлечениях, разработка и исследование методов и алгоритмов анализа текста, устной речи и изображений>>.

{\theorinfluence} состоит в постановке задачи анализа представленности наборов генов для экспериментов со сложным биологическим дизайном, разработке методов, решающих задачу как для случая простого дизайна, так и сложного.

{\influence} состоит в том, что реализованные методы могут быть использованы для упрощения процесса анализа и изучения биологических процессов, протекающих в живых организмах. 
Также они позволяют исследователям синтезировать абсолютно новые предположения о биологических закономерностях, и находить подтверждения для гипотез, сформированных раннее. 

\fixme{{\methods} \ldots}


% В папке Documents можно ознакомиться с решением совета из Томского~ГУ
% (в~файле \verb+Def_positions.pdf+), где обоснованно даются рекомендации
% по~формулировкам защищаемых положений.

\fixme{{\reliability} полученных результатов обеспечивается \ldots \ Результаты находятся в соответствии с результатами, полученными другими авторами.}

{\integration} Разработанный метод FGSEA-Multilevel был реализован и внедрен в пакет fgsea, написанный на языках программирования R и C++. 
Данный пакет является свободно доступным и размещен в архиве биологических пакетов Bioconductor (\url{https://www.bioconductor.org/}).
При этом пакет fgsea входит в число наиболее популярных пакетов из всех доступных пакетов в архиве.
Так в настоящее время ранг пакета по числу обращений и скачиваний равен 38-ми из более чем 2000 различных пакетов, представленных в Bioconductor.
Разработанный метод GESECA также является свободно доступным и может быть скачан с github-репозитория по адресу \url{https://github.com/ctlab/fgsea/tree/geseca}. Поисковая система релевантных биологических экспериментов по заданному набору генов реализована в виде веб-сервиса и доступна по адресу \url{https://ctlab.itmo.ru/geseca/}. Все разработанные методы активно применяются в процессе подготовки магистров по программе <<Биоинформатика и системная биология>> в Университете ИТМО.


{\probation}
Основные результаты работы докладывались на:
\begin{enumerate}
    \item Всероссийский конгресс молодых ученых Университета ИТМО, 2019 г., Университет ИТМО, Санкт-Петербург.
    \item Всероссийская научная конференция по проблемам информатики СПИСОК-2019. 2019 г., СПбГУ, Санкт-Петербург.
    \item Moscow Conference on Computational Molecular Biology (MCCMB 2019), 2019 г., Москва.
    \item Всероссийский конгресс молодых ученых Университета ИТМО, 2020 г., Университет ИТМО, Санкт-Петербург
    \item CSHL Meeting on Biological Data Science 2020,  2020 г., онлайн конференция.
    \item Moscow Conference on Computational Molecular Biology (MCCMB 2021). 2021 г., Москва.
    \item Научная и учебно-методическая конференция Университета ИТМО, 2022 г., Университет ИТМО, Санкт-Петербург.
\end{enumerate}

{\contribution} Автор принимал активное участие в процессе анализа и совершенствования метода FGSEA-Multilevel. 
Автором были определены необходимые параметры для сходимости метода при приложениях и применениях метода на практике. 
Был осуществлен анализ теоретических свойств распределения P-значений и совершен корректный переход к вычислениям логарифмов P-значений.
Автор принимал активное участие во всех этапах, связанных с разработкой и реализацией метода GESECA, совместно с Сергушичевым А.А. и Артемовым М.Н.
Диссертантом была полностью написана программная реализация метода GESECA.
Проведены валидация и верификация корректности работы метода на практике.
Автором разработан веб-сервис, который служит поисковой системой для поиска релевантных экспериментов по заданному набору генов.

\ifnumequal{\value{bibliosel}}{0}
{%%% Встроенная реализация с загрузкой файла через движок bibtex8. (При желании, внутри можно использовать обычные ссылки, наподобие `\cite{vakbib1,vakbib2}`).
    {\publications} Основные результаты по теме диссертации изложены
    в~XX~печатных изданиях,
    X из которых изданы в журналах, рекомендованных ВАК,
    X "--- в тезисах докладов.
}%
{%%% Реализация пакетом biblatex через движок biber
    \begin{refsection}[bl-author, bl-registered]
        % Это refsection=1.
        % Процитированные здесь работы:
        %  * подсчитываются, для автоматического составления фразы "Основные результаты ..."
        %  * попадают в авторскую библиографию, при usefootcite==0 и стиле `\insertbiblioauthor` или `\insertbiblioauthorgrouped`
        %  * нумеруются там в зависимости от порядка команд `\printbibliography` в этом разделе.
        %  * при использовании `\insertbiblioauthorgrouped`, порядок команд `\printbibliography` в нём должен быть тем же (см. biblio/biblatex.tex)
        %
        % Невидимый библиографический список для подсчёта количества публикаций:
        \printbibliography[heading=nobibheading, section=1, env=countauthorvak,          keyword=biblioauthorvak]%
        \printbibliography[heading=nobibheading, section=1, env=countauthorwos,          keyword=biblioauthorwos]%
        \printbibliography[heading=nobibheading, section=1, env=countauthorscopus,       keyword=biblioauthorscopus]%
        \printbibliography[heading=nobibheading, section=1, env=countauthorconf,         keyword=biblioauthorconf]%
        \printbibliography[heading=nobibheading, section=1, env=countauthorother,        keyword=biblioauthorother]%
        \printbibliography[heading=nobibheading, section=1, env=countregistered,         keyword=biblioregistered]%
        \printbibliography[heading=nobibheading, section=1, env=countauthorpatent,       keyword=biblioauthorpatent]%
        \printbibliography[heading=nobibheading, section=1, env=countauthorprogram,      keyword=biblioauthorprogram]%
        \printbibliography[heading=nobibheading, section=1, env=countauthor,             keyword=biblioauthor]%
        \printbibliography[heading=nobibheading, section=1, env=countauthorvakscopuswos, filter=vakscopuswos]%
        \printbibliography[heading=nobibheading, section=1, env=countauthorscopuswos,    filter=scopuswos]%
        %
        \nocite{*}%
        %
        {\publications} Основные результаты по теме диссертации изложены в~\arabic{citeauthor}~печатных изданиях,
        \arabic{citeauthorvak} из которых изданы в журналах, рекомендованных ВАК\sloppy%
        \ifnum \value{citeauthorscopuswos}>0%
            , \arabic{citeauthorscopuswos} "--- в~периодических научных журналах, индексируемых Web of~Science и Scopus\sloppy%
        \fi%
        \ifnum \value{citeauthorconf}>0%
            , \arabic{citeauthorconf} "--- в~тезисах докладов.
        \else%
            .
        \fi%
        \ifnum \value{citeregistered}=1%
            \ifnum \value{citeauthorpatent}=1%
                Зарегистрирован \arabic{citeauthorpatent} патент.
            \fi%
            \ifnum \value{citeauthorprogram}=1%
                Зарегистрирована \arabic{citeauthorprogram} программа для ЭВМ.
            \fi%
        \fi%
        \ifnum \value{citeregistered}>1%
            Зарегистрированы\ %
            \ifnum \value{citeauthorpatent}>0%
            \formbytotal{citeauthorpatent}{патент}{}{а}{}\sloppy%
            \ifnum \value{citeauthorprogram}=0 . \else \ и~\fi%
            \fi%
            \ifnum \value{citeauthorprogram}>0%
            \formbytotal{citeauthorprogram}{программ}{а}{ы}{} для ЭВМ.
            \fi%
        \fi%
        % К публикациям, в которых излагаются основные научные результаты диссертации на соискание учёной
        % степени, в рецензируемых изданиях приравниваются патенты на изобретения, патенты (свидетельства) на
        % полезную модель, патенты на промышленный образец, патенты на селекционные достижения, свидетельства
        % на программу для электронных вычислительных машин, базу данных, топологию интегральных микросхем,
        % зарегистрированные в установленном порядке.(в ред. Постановления Правительства РФ от 21.04.2016 N 335)
    \end{refsection}%
    \begin{refsection}[bl-author, bl-registered]
        % Это refsection=2.
        % Процитированные здесь работы:
        %  * попадают в авторскую библиографию, при usefootcite==0 и стиле `\insertbiblioauthorimportant`.
        %  * ни на что не влияют в противном случае
        \nocite{vakbib2}%vak
        \nocite{patbib1}%patent
        \nocite{progbib1}%program
        \nocite{bib1}%other
        \nocite{confbib1}%conf
    \end{refsection}%
        %
        % Всё, что вне этих двух refsection, это refsection=0,
        %  * для диссертации - это нормальные ссылки, попадающие в обычную библиографию
        %  * для автореферата:
        %     * при usefootcite==0, ссылка корректно сработает только для источника из `external.bib`. Для своих работ --- напечатает "[0]" (и даже Warning не вылезет).
        %     * при usefootcite==1, ссылка сработает нормально. В авторской библиографии будут только процитированные в refsection=0 работы.
}

При использовании пакета \verb!biblatex! будут подсчитаны все работы, добавленные
в файл \verb!biblio/author.bib!. Для правильного подсчёта работ в~различных
системах цитирования требуется использовать поля:
\begin{itemize}
        \item \texttt{authorvak} если публикация индексирована ВАК,
        \item \texttt{authorscopus} если публикация индексирована Scopus,
        \item \texttt{authorwos} если публикация индексирована Web of Science,
        \item \texttt{authorconf} для докладов конференций,
        \item \texttt{authorpatent} для патентов,
        \item \texttt{authorprogram} для зарегистрированных программ для ЭВМ,
        \item \texttt{authorother} для других публикаций.
\end{itemize}
Для подсчёта используются счётчики:
\begin{itemize}
        \item \texttt{citeauthorvak} для работ, индексируемых ВАК,
        \item \texttt{citeauthorscopus} для работ, индексируемых Scopus,
        \item \texttt{citeauthorwos} для работ, индексируемых Web of Science,
        \item \texttt{citeauthorvakscopuswos} для работ, индексируемых одной из трёх баз,
        \item \texttt{citeauthorscopuswos} для работ, индексируемых Scopus или Web of~Science,
        \item \texttt{citeauthorconf} для докладов на конференциях,
        \item \texttt{citeauthorother} для остальных работ,
        \item \texttt{citeauthorpatent} для патентов,
        \item \texttt{citeauthorprogram} для зарегистрированных программ для ЭВМ,
        \item \texttt{citeauthor} для суммарного количества работ.
\end{itemize}
% Счётчик \texttt{citeexternal} используется для подсчёта процитированных публикаций;
% \texttt{citeregistered} "--- для подсчёта суммарного количества патентов и программ для ЭВМ.

Для добавления в список публикаций автора работ, которые не были процитированы в
автореферате, требуется их~перечислить с использованием команды \verb!\nocite! в
\verb!Synopsis/content.tex!.
